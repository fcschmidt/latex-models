%%%%%%%%%%%%%%%%%%%%%%%%%%%%%%%%%%%%%%%%%
% APS - Atividade Pr�tica Supervisionada
% LateX Template
% Version 1.0 (19/06/2017)
%
% Author:
% F�bio C. Schmidt (fcschimidt@gmail.com)
%
% License:
% CC BY-NC-SA 3.0 (http://creativecommons.org/licenses/by-nc-sa/3.0/)
%
% Nota:
% Use o compiler.sh para compilar e excluir aquivos gerados pela compila��o
%
%%%%%%%%%%%%%%%%%%%%%%%%%%%%%%%%%%%%%%%%%

%----------------------------------------------------------------------------------------
%	PACOTES E OUTRAS CONFIGURA��ES DO DOCUMENTO
%----------------------------------------------------------------------------------------
\documentclass[12pt]{article}
\usepackage{styles}

  \hypersetup
  {
    pdftitle = {Titulo da Tese},
    pdfauthor = {Nome do autor},
    pdfsubject = {Assunto},
    pdfkeywords = {palavras, chave},
    pdfcreator = {LaTeX},
    linkbordercolor = {1 1 1},
    pdftoolbar = true,
    pdfmenubar = true,
    colorlinks = true,
    citecolor=black,
    linkcolor=black,
    urlcolor=black,
    %filecolor=blue,  
	%	urlcolor=blue,
	%	bookmarksdepth=4
  }
  
  \makeindex 
  %----------------------------------------------------------------------------------------
  
  %----------------------------------------------------------------------------------------
  %	CONTE�DO DO DOCUMENTO
  %----------------------------------------------------------------------------------------
  \begin{document}
  
  %----------------------------------------------------------------------------------------
  %	APS CAPA CONTE�DO
  %----------------------------------------------------------------------------------------
  %-------------------------------------------------
%     CAPA
%-------------------------------------------------
\begin{titlepage}
  \centering
  \Large
  \textbf{Instituto de Ciências Exatas e Tecnologia \\ 
  			Universidade Paulista - UNIP}\\
  Campus Araraquara, SP\\[2em]
  
  Proposta de projeto\\[7em]
  \textsc{\textbf{\fontsize{17}{\baselineskip}\selectfont Estratégia de Navegação em Róbotica Móvel para Exploração e Patrulhamento de um Ambiente}}\\[7em]

  Aluno: \\ Fábio César Schimidt \\[2em]
  Orientador: \\ Prof. Dr. Leandro Carlos Fernandes\\[9em]
  Araraquara, Março de 2017
\end{titlepage}
  %----------------------------------------------------------------------------------------

  \setcounter{page}{1} % Adicionando a numera��o de p�gina para o restante do documento
  \pagestyle{plain} % estilo da p�gina

  \newpage

  \tableofcontents
  \newpage
  
  %----------------------------------------------------------------------------------------
  %	APS CONTE�DO DOS ARQUIVOS
  %	Cada se��o � importada separadamente, 
  % abra cada arquivo por vez para modificar o conte�do
  %----------------------------------------------------------------------------------------
  %-------------------------------------------------
%    OBJETIVO e MOTIVAÇÃO DO TRABALHO
%-------------------------------------------------
\section{Objetivo e Motivação do Trabalho}
\textbf{Aqui vai os objetivo e a Motivação para fazer esse trabalho.}


Lorem ipsum dolor sit amet, consectetur adipiscing elit. Pellentesque at pretium
diam. Praesent ut sollicitudin nisl. Etiam vehicula convallis leo, vel pulvinar tortor
volutpat a. Duis tempus felis nec arcu sagittis, ac tincidunt urna molestie. Cras quis
tristique libero, ac posuere nisl. Nam commodo id felis a molestie. Phasellus tincidunt
molestie commodo. Fusce efficitur ullamcorper elit sit amet dapibus. Nullam semper
ligula vel lectus vehicula finibus.
  %--------------------------------------------------
%    INTRODUÇÃO
%--------------------------------------------------
\section{Introdução}
\textbf{Uma breve introdução e contextualização do trabalho.}

Lorem ipsum dolor sit amet, consectetur adipiscing elit. Pellentesque at pretium
diam. Praesent ut sollicitudin nisl. Etiam vehicula convallis leo, vel pulvinar tortor
volutpat a. Duis tempus felis nec arcu sagittis, ac tincidunt urna molestie. Cras quis
tristique libero, ac posuere nisl. Nam commodo id felis a molestie. Phasellus tincidunt
molestie commodo. Fusce efficitur ullamcorper elit sit amet dapibus. Nullam semper
ligula vel lectus vehicula finibus.
  %--------------------------------------------------
%    FUNDAMENTOS 
%--------------------------------------------------
\section{Fundamentos}
\textbf{Aqui será adicionado os fundamentos do trabalho juntamente com conceitos gerais.}

Lista:
\begin{itemize}
\item \textbf{Item 1:} Lorem ipsum dolor sit amet, consectetur adipiscing elit. Pellentesque at pretium diam. 
\item \textbf{Item 2:} Lorem ipsum dolor sit amet, consectetur adipiscing elit. Pellentesque at pretium diam.
\item \textbf{Item 3:} Lorem ipsum dolor sit amet, consectetur adipiscing elit. Pellentesque at pretium diam.
\item \textbf{Item 4:} Lorem ipsum dolor sit amet, consectetur adipiscing elit. Pellentesque at pretium diam.
\item \textbf{Item 5:} Lorem ipsum dolor sit amet, consectetur adipiscing elit. Pellentesque at pretium diam.
\end{itemize}

Lorem ipsum dolor sit amet, consectetur adipiscing elit. Pellentesque at pretium
diam. Praesent ut sollicitudin nisl. Etiam vehicula convallis leo, vel pulvinar tortor
volutpat a. Duis tempus felis nec arcu sagittis, ac tincidunt urna molestie. Cras quis
tristique libero, ac posuere nisl. Nam commodo id felis a molestie. Phasellus tincidunt
molestie commodo. Fusce efficitur ullamcorper elit sit amet dapibus. Nullam semper
ligula vel lectus vehicula finibus.
  \input{arquivos/plano-dev.tex}
  %--------------------------------------------------
%    PROJETO Do PROGRAMA
%--------------------------------------------------
\section{Projeto do Programa}
\textbf{Nesta seção será descrita toda à estrutura e módulos que serão desenvolvidos do programa.\\
É recomendavel quebrar em subseções para descrever essas estruturas.\\
Segue um exemplo abaixo:}


\label{sec::projeto}
\subsection{Classes}
\begin{itemize}
\item \textbf{Classe RecebeValor:} Lorem ipsum dolor sit amet, consectetur adipiscing elit. Pellentesque at pretium diam. 
\item \textbf{Classe ArmazenaValor:} Lorem ipsum dolor sit amet, consectetur adipiscing elit. Pellentesque at pretium diam. 
\item \textbf{Classe ImprimeValor:} Lorem ipsum dolor sit amet, consectetur adipiscing elit. Pellentesque at pretium diam.
\end{itemize}

\newpage
\subsection{Principais métodos da Classe RecebeValor}
\begin{itemize}
\item \textbf{Input:} Lorem ipsum dolor sit amet, consectetur adipiscing elit. Pellentesque at pretium diam;
\item \textbf{Output:} Lorem ipsum dolor sit amet, consectetur adipiscing elit. Pellentesque at pretium diam;
\item \textbf{Calcula:} Lorem ipsum dolor sit amet, consectetur adipiscing elit. Pellentesque at pretium diam;
\end{itemize}


\subsection{Principais métodos da Classe ArmazenaValor}
\begin{itemize}
\item \textbf{Insert:} Lorem ipsum dolor sit amet, consectetur adipiscing elit. Pellentesque at pretium diam;
\item \textbf{Update:} Lorem ipsum dolor sit amet, consectetur adipiscing elit. Pellentesque at pretium diam;
\item \textbf{Delete:} Lorem ipsum dolor sit amet, consectetur adipiscing elit. Pellentesque at pretium diam;
\end{itemize}


\subsection{Principais métodos da Classe ImprimeValor}
\begin{itemize}
\item \textbf{Imprimir:} Lorem ipsum dolor sit amet, consectetur adipiscing elit. Pellentesque at pretium diam;
\end{itemize}
  %---------------------------------------------------
%    RELATÓRIO COM LINHAS DE CÓDIGO DO PROGRAMA
%---------------------------------------------------
\section{Relatório com linhas de código do programa}
\begin{lstlisting}[language=html]
<html>
    <head>
        <title>Hello</title>
    </head>
    <body>Hello</body>
</html>
\end{lstlisting}


%------- 
% Classes RecebeValor
%-------
\subsection{Classes RecebeValor}
\subsubsection{Input}
\subsubsection{Output}
\subsubsection{Calcula}

%------- 
% Classes ArmazenaValor
%-------
\subsection{Classes ArmazenaValor}
\subsubsection{Insert}
\subsubsection{Update}
\subsubsection{Delete}

%------- 
% Classes ImprimeValor
%-------
\subsection{Classes RecebeValor}
\subsubsection{Imprime}
  %--------------------------------------------------
%    APRESENTAÇÃO DO PROGRAMA
%--------------------------------------------------
\section{Apresentação do programa em Funcionamento}
\textbf{Apresentação do programa em funcionamento em um computador, apresentando todas as funcionalidades pedidas e extras.}

\textit{
Lorem ipsum dolor sit amet, consectetur adipiscing elit. Pellentesque at pretium
diam. Praesent ut sollicitudin nisl. Etiam vehicula convallis leo, vel pulvinar tortor
volutpat a. Duis tempus felis nec arcu sagittis, ac tincidunt urna molestie. Cras quis
tristique libero, ac posuere nisl. Nam commodo id felis a molestie. Phasellus tincidunt
molestie commodo. Fusce efficitur ullamcorper elit sit amet dapibus. Nullam semper
ligula vel lectus vehicula finibus.
}

\begin{figure}
\centering
	\includegraphics[width=0.90\textwidth]{imagens/programa.jpg}
	\caption{Seu programa rodando.}
\end{figure}
  \newpage
  \section{Referências}
\nocite{*}
\vspace{-3em}
\bibliographystyle{plain}
\renewcommand{\refname}{} % define como nul nome da seção
\bibliography{referencias}
\thispagestyle{empty}
  %----------------------------------------------------------------------------------------
  
  \end{document}
  %---------------------------------------------------------------------------------------